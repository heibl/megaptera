
\documentclass[12pt]{article}
\usepackage{geometry} % see geometry.pdf on how to lay out the page. There's lots.
\geometry{a4paper} % or letter or a5paper or ... etc
% \geometry{landscape} % rotated page geometry

\usepackage{natbib}

%%% Colors
\usepackage{color}
\definecolor{red}{rgb}{1, 0, 0}
\definecolor{dred}{rgb}{.5, 0, 0}
\definecolor{blue}{rgb}{0, 0, .5}
\definecolor{lgreen}{rgb}{.68, .87, .55}
\definecolor{dgreen}{rgb}{0, .27, .16}
\definecolor{grey}{rgb}{.3, .3, .3}
%% Hyperlinks
\usepackage[colorlinks=true,
            linkcolor=dred,
            citecolor=blue,
            urlcolor=blue]{hyperref}
            
\newcommand{\acctab}{table \texttt{acc\_$<$locus$>$ }}

\title{MEGAPTERA internal description}
\author{Christoph Heibl\\Harztalstr.~29,\\83714 Miesbach, Germany}
%\date{} % delete this line to display the current date

%%% BEGIN DOCUMENT
\begin{document}

\maketitle
\tableofcontents

\newpage
\section{Description of \texttt{step*} functions}

\subsection{stepA}

\begin{enumerate}
\item 
\end{enumerate}

%: B
\subsection{stepB}

\begin{enumerate}
\item If \texttt{update.seqs = "all"} and if \acctab exists, i.e.~if \texttt{stepB} has been run before, delete \acctab thereby triggering a thoroughly new search.

\item (Re-)create \acctab.

\item Create a list of taxon names to be passed to \texttt{downloadSequences} (either serial or parallel). All downloaded sequences will be written to \acctab with the attribute \texttt{status} set to \texttt{`raw'}.

\item Search the attribute \texttt{spec\_ncbi} for a set of regular expressions indicating sequences that stem from samples that are undetermined at the level of species and set their attribute \texttt{status} to \texttt{`excluded (indet)'}.

\item Crop subspecies names in attribute \texttt{taxon} with \texttt{strip.infraspec()}; the full names are still available in attribute \texttt{spec\_ncbi}.

\item Exclude sequences that are too long to align, i.e., sequences exceeding the the number of \texttt{max.bp} base pairs (default is 5000 bp), by tagging their attribute \texttt{status} as \texttt{`excluded (too long)'}.

\item Run \texttt{dbMaxGIPerSpec} to chose the \texttt{max.gi.per.spec} longest sequences per species for alignment; the rest will be tagged as \texttt{`excluded (max.gi)'}.

\item Run \texttt{dbUpdateTaxonomy} to detect species with sequences that have no entry in the table \texttt{taxonomy}. Sequences of species that cannot be classified are tagged as \texttt{`excluded (unclassified)'} in the attribute \texttt{status}.

\item Issue summary on screen and exit.
\end{enumerate}

%: C
\subsection{stepC}

\begin{enumerate}
\item Check if table \texttt{acc\_*} exists, i.e.~if \texttt{stepB} has been run. If not, exit with error.
\item Clear results from previous runs of \texttt{stepD}.
\item Clear results from previous runs of \texttt{stepE}.


\item Produce table of species counting numbers of sequences and assessing if species are aligned with the \texttt{char\_length() function}. If the table is empty, exit without error.

\item Mark single-sequence species in the \texttt{status} column with \texttt{`single'}.

\end{enumerate}

\subsection{stepD}

\begin{enumerate}
\item 
\end{enumerate}

\subsection{stepE}

\begin{enumerate}
\item 
\end{enumerate}



\subsection{stepF}

\begin{enumerate}
\item Set threshold values for \texttt{min.identity} and \texttt{min.coverage}.
\item Open database connection.
\item Check if \texttt{stepE} has been run; if not stop.
\item Check if \texttt{stepF} has been run before, i.e. if MSA table exists. YES: go to 5. NO: go to xx.
\item Check if MSA table needs to be updated. This implies checking if the set of species names has changed, but also if the set of GIs has changed in any species. Maybe md5-checksums could be used to achieve this?
\item Erase downstream results before updating: \texttt{spec/gen\_gene}, nexus and phylip files.
\end{enumerate}

\subsection{stepG}

\begin{enumerate}
\item Check if MSA table exists. YES: go to next step. NO: break.
\item Check if any entry in the \texttt{status} column equals \texttt{`raw'}, which is the trigger for aligning the sequences in the MSA table. 
\end{enumerate}

%% DATABASE
\section{Description of database}

%% TAXONOMY
\subsection{Table \texttt{taxonomy}}

%% REFERENCE
\subsection{Table \texttt{reference}}

%% ACC
\subsection{Table \texttt{acc\_*}}

\begin{description}
\item[gi] 
\item[taxon] 
\item[spec\_ncbi] 
\item[status] describes the status of the sequence along the pipeline; xx values are defined and are listed in the order of their appearence along the pipeline:
	\begin{description}
	\item[\texttt{`raw'}] is the default status for every downloaded sequence.
	\item[\texttt{`excluded (indet)'}] is set by \texttt{stepB}.
	\item[\texttt{`excluded (too long)'}] is set by \texttt{stepB} or \texttt{stepF}.
	\item[\texttt{`excluded (max.gi)'}] is set by \texttt{stepB}.
	\item[\texttt{`excluded (unclassified)'}]  is set by \texttt{dbUpdateTaxonomy}. For species names that are present in any one \texttt{acc\_*} table but not in the \texttt{taxonomy} table, this function tries to derive their taxonomic classification using congenerics. If this is not possible, a species (and all its sequences) are tagged as unclassified and excluded from downstream steps of the pipeline.
	\item[\texttt{`single'}] is set by \texttt{stepC} and marks all species that are represented with only one sequence, not requiring alignment.
	\item[\texttt{`aligned'}] is set by \texttt{stepC} for all aligned species.
	\end{description}

\item[genom] 
\item[npos] 
\item[identity] 
\item[coverage] 
\item[dna]
\end{description}


%% SPEC/GEN
\subsection{Table \texttt{spec\_*}/\texttt{gen\_*}}

\begin{description}
\item[ ] 
\end{description}


%% STATUS
\section{Checking pipeline status}

%% UPDATE
\section{Updating the database}

The pipeline is designed to minimize computational costs when updating a \textsc{megaptera} project.
\vspace{\baselineskip}


%\begin{table}
\begin{tabular}{lll}
\hline
Step&Changes&Trigger\\
\hline
A&&\\
\hline
B&new sequences&none; GenBank has to be searched\\
\hline
C&new sequences&conspecifics of unequal length\\
\hline
D&&\\
\hline
E&&\\
\hline
F&&\\
\hline
G&new sequences&\texttt{`raw'} in \texttt{status} column\\
 &new guide tree&\emph{not implemented}\\
  &species excluded&\texttt{`raw'} in \texttt{status} column\\
\hline
H&(re)alignment by G&\texttt{`aligned'} in \texttt{status} column\\
\hline
I&&\\
\hline
\end{tabular}
%\end{table}

\paragraph{Problems}
\begin{itemize}
\item \texttt{stepC} deletes entries the attributes \texttt{identity} and \texttt{coverage} of the \texttt{acc\_*} table. This triggers the rerunning of \texttt{stepE}.
\end{itemize}


%% PARALLELIZATION
\section{Parallelization}

These functions contain parallelized apply-like functions:

\begin{itemize}
    \item \texttt{ncbiLineage}
\end{itemize}

%% EXTENDED INGROUP / SURROGATES
\section{Extended ingroup and surrogate species}





%% Bibliography
\bibliography{/Users/stoffi/Documents/latex/bib/biology}
\bibliographystyle{/Users/stoffi/Documents/latex/bst/sysbio.bst}

\end{document}
